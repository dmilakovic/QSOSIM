\documentclass[a4paper,12pt]{article}
%\usepackage{graphicx}
%\usepackage{color}
%\usepackage{wrapfig}
\usepackage{placeins}
\usepackage{setspace} 
\usepackage[margin=1.5cm]{geometry}
\setlength\parindent{0pt}
\renewcommand\rmdefault{ptm} 
\newcommand{\tab}{\hspace*{2em}}

\begin{document} 
\title{Notes on the program QSOSIM9 - John Webb, Dec 2013}
\date{\small {\today}}
\maketitle

This program generates simulated spectra including the forest and
including up to 20 additional user-specified absorption systems
(presumably DLAs or LLSs).  It only uses HI at present.
Writes normalised spectrum to a new ascii file,
{\it spec.dat}, 4-column format: $\lambda$, flux, $\sigma$, no-noise-flux.\\

{\bf Random selection of redshifts for forest:}

QSOSIM9 uses the simplest possible redshift evolution for the forest.  Lines 
are unclustered and the number of lines per unit redshift is taken as
\[
dn/dz = A(1+z)^{\gamma}
\]
$\gamma$ is a user-input via {\it sim.dat} (see below).  $A$ is the
normalisation measured for a specific detection threshold (and is hard coded).
Penton et al 2004 find $\gamma = 1.85$ for $z>2.5$ and $\gamma = 0.16$ for 
$z<2.5$, so note the approximation used in QSOSIM9.  The default in {\it sim.dat} 
is $\gamma = 2$.  Integrating to get the total number of lines detected in the 
range $z_1 - z_2$,
\[
n = \frac{A}{\gamma +1}\left[ (1+z_2)^{\gamma +1} - ((1+z_1)^{\gamma +1})\right]
\]
The cumulative probability function is therefore
\[
c = \frac{(1+z)^{\gamma +1} - ((1+z_1)^{\gamma +1}}{(1+z_2)^{\gamma +1} - 
((1+z_1)^{\gamma +1}} 
\]
for $z_1 < z < z_2$, i.e. $c$ is a uniform random variable, $0 < c < 1$.

Let $p = (1+z_2)^{\gamma +1}$, $q = (1+z_1)^{\gamma +1}$.  Then
\[
c(p-q)+q = (1+z)^{\gamma +1}
\]
\[
\left[ \frac{\log_{10}(c(p-q)+q)}{\gamma +1} \right] = \log_{10}(1+z) = x
\]
Re-arranging for $z$,
\[
z = 10^x -1
\]

{\bf Random selection of $\boldmath N_{HI}$ for forest:}

A single power law distribution is used, with lower cutoff.
The number of lines per unit column density interval is
\[
\frac{dn}{dN} \propto N^{-\beta}
\]
where $\beta = 1.7$ is currently hard-coded.
Note Penton et al 2004 suggest
$\beta = 1.65 \pm 0.007$ for $10^{12.5} < N < 10^{14.5}$ and a flatter 
slope for $N > 10^{14.5}$.

Use the same procedure as above, i.e.
integrate and form the cumulative probability function again,
\[
d = \frac{N_c^{1-\beta} - N^{1-\beta}}{N_c^{1-\beta} - N_{hi}^{1-\beta}}
\]
where $d$ is a uniform random variable, $0 < d < 1$,
$N_c$ and $N_{hi}$ are the low and high cut-offs.
$N_c = 3e13$ is the default value in {\it sim.dat}.

\newpage

To a good approximation (since $N_{hi}^{1-\beta} \sim 10^{22}$), 
this is
\[
d = \frac{N_c^{1-\beta} - N^{1-\beta}}{N_c^{1-\beta}}
\]
This re-arranges for $N$ to give
\[
\log_{10}(N) = \frac{\log_{10}(1-d)}{1-\beta} + \log_{10}(N_c)
\]

{\bf Random selection of $b$ for forest:}

This is a straightforward Gaussian, with default mean and standard deviation
23 and 3 km/s. \\

{\bf Random number generators:}

{\it ran3} (Numerical Recipes) is used for uniform random numbers.  
Don't use {\it ran1} - it doesn't seem to work with gfortran on a Mac. \\

{\bf Error array:}

There are 2 simple models.  One produces a constant S/N per pixel in
the continuum (no wavelength dependence) but uses a base-value
at zero-flux levels (parameter inoise=0 in {\it sin.dat}).
For this model, the 1-$\sigma$ error array, $\sigma$, is hard-coded as
\[
\sigma = \frac{I_{obs}}{s/n} + 0.2\frac{I_0}{s/n}
\]
where $s/n$ is the signal-to-noise parameter given in {\it sim.dat},
$I_{obs}$ is the observed intensity, $I_0$ is the unabsorbed intensity,
so note that the ``actual'' signal-to-noise is not equal to the number
entered in {\it sim.dat}. The reason for doing this (and not simply using
$\sqrt n$ statistics) is saturated lines and Lyman limits would remain
noise-free.\\

The other model (parameter inoise=1 in {\it sin.dat}) degrades the noise 
towards the blue, using:
\[
noise \propto (1+e^{-x})
\]
where
\[
x = \frac{\lambda - c_1}{c_2}
\]
and where default (hard-coded) values are $c_1 = 3532$ and $c_2 = 117$.\\

\newpage

{\bf Input file {\it sin.dat}:}\\

\begin{tabular}{ l l }
3.0 & Quasar emission redshift.\\
-0.7 & Spectral index.\\
16.0 & V magnitude.\\
3000 & Starting wavelength.\\
5200 & End wavelength.\\
0.03 & Pixel size.\\
3.0e12 & N(HI) lower cut-off for forest.  No forest lines below this value included.\\
1.00e16 & N(HI) upper limit for forest.  No forest lines above this value included.\\
3.0 & Spectral resolution, sigma, km/s.\\
100 & Signal-to-noise ratio per pixel (but see above).\\
1 & Parameter {\it inoise} in the code.  See above.\\
2 & Number of user-specified additional absorption systems (e.g. DLAs).\\
  & You can specify 0 if you want.  Max. coded is 20.\\
100 & Avoidance zone around each additional system in km/s.\\
1.0e21 & N(HI) of first system.\\
10.0 & b-parameter of first system.\\
2.86618079 & Redshift of first system.\\
1.0e21 & N(HI) of second system.\\
10.0 & b-parameter of second system.\\
2.86618079 & Redshift of second system.\\
etc. &\\
\end{tabular}

\begin{thebibliography}{99}

\bibitem{Penton04} Penton, Stocke, Shull, ApJ Suppl., 152, 29, 2004

\end{thebibliography}

\end{document}
